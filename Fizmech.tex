% "Станет проще"

\documentclass[letterpaper,12pt]{article} % тип документа

% report, book

% Рисунки
\usepackage{graphicx}
\usepackage{wrapfig}

\usepackage{hyperref}
\usepackage[rgb]{xcolor}
\hypersetup{				% Гиперссылки
    colorlinks=true,       	% false: ссылки в рамках
	urlcolor=blue          % на URL
}

%  Русский язык

\usepackage[T2A]{fontenc}			% кодировка
\usepackage[utf8]{inputenc}			% кодировка исходного текста
\usepackage[english,russian]{babel}	% локализация и переносы


% Математика
\usepackage{amsmath,amsfonts,amssymb,amsthm,mathtools} 


\usepackage{wasysym}

\begin{document}

\title{Исследование ионизационного равновесия газа}
\author{В. Дочкина}
\date{Ноябрь 2018}
\maketitle

\begin{abstract}
Данная работа посвящена изучению и расчету ионизационного равновесия в газе. Для расчета используются методы статистической физики, в частности формула Саха (Saha, 1920). 
Рассмотрена равновесная (термическая) ионизация атомов Ne и He в приближении идеального газа. 
\end{abstract}


\section{Введение}

Представлено теоретическое и математическое обоснование расчетных формул и принятых предположений. 
Расчет проводится для температур от 3 000 до 30 000 градусов Кельвина, в диапазоне давлений от 0,1 до 1 атмосферы. 
В качестве результата представлена зависимость степени ионизации от температуры и давления.

\section{Теория}
Реакция ионизации:
\begin{equation} \label{eq:aperp} % the label is used to reference the equation
A\rightleftharpoons A^{+}+e^{-}
\end{equation}
ЗСМ:
\begin{equation} \label{eq:aperp} % the label is used to reference the equation
m_{a}=m_{i}+m_{e}
\end{equation}
ЗСЭ:
\begin{equation} \label{eq:aperp} % the label is used to reference the equation
E_{0}^{a}=E_{0}^{i}+E_{0}^{e}-I ;§
\end{equation}
где I - потенциал ионизации
\\*
Так как
\begin{equation} \label{eq:aperp} % the label is used to reference the equation
m_{a}\cong m_{i}
\end{equation}
То
\begin{equation} \label{eq:aperp} % the label is used to reference the equation
\lambda_{a}\cong \lambda_{i}
\end{equation} 
Элементарный баланс энергии:
\begin{equation} \label{eq:aperp} % the label is used to reference the equation
\mu_{a}=\mu_{i}+\mu_{e}
\end{equation}
Длина волны де Бройля:
\begin{equation} \label{eq:aperp} % the label is used to reference the equation
\lambda_{b}= \frac{2\pi\hbar}{\sqrt{2\pi m T}}
\end{equation}
Химический потенциал:
\begin{equation} \label{eq:aperp} % the label is used to reference the equation
\mu_{a}=\frac{\partial F_{a}}{\partial N_{a}} ;
\end{equation}
\begin{equation} \label{eq:aperp} % the label is used to reference the equation
F= -Tln{Z} 
\end{equation}
- где F - cвободная энергия; 
\begin{equation} \label{eq:aperp} % the label is used to reference the equation
Z = \frac{z^n}{N!}; z = z_{tr}*z_{in} 
\end{equation}
Используя:
\begin{equation} \label{eq:aperp} % the label is used to reference the equation
F = -Tln{Z}=-Tln{\frac{z^N}{N!}} 
\end{equation}
Получаем:
\begin{equation} \label{eq:aperp} % the label is used to reference the equation
ln{N!}=NlnN-N 
\end{equation}
\begin{equation} \label{eq:aperp} % the label is used to reference the equation
F=T(ln{z^N} - ln{N!})=-T[-NlnN + Nlnz + Nln{e}]= - TNln{\frac{ez}{N}}
\end{equation}
 \begin{equation} \label{eq:aperp} % the label is used to reference the equation
\mu_{a}=\frac{\partial F_a}{\partial N_a}=-T \frac{\partial}{\partial N_a}[\sum N_a ln{\frac{ez_a}{N_a}}]=-T\frac{\partial}{\partial N_a}[\sum N_a ln{\frac{e \not{V} z_{in}}{\lambda{_a}^3_{b}n_a \not{V} }}] 
\end{equation}
\begin{equation} \label{eq:aperp} % the label is used to reference the equation
z_{post}=\frac{V}{\lambda^3_{b}}; n_a=\frac{N_a}{V} 
\end{equation}
\begin{equation} \label{eq:aperp} % the label is used to reference the equation
\mu_a=-T ln{\frac{z_{in}}{\lambda^3_a n_A}}= T ln{\frac{\lambda^3_a n_a}{z{_a}_{in}}} 
\end{equation}
Обозначим:
\begin{equation} \label{eq:aperp} % the label is used to reference the equation
E_{j}-E_{0}=E_{j}^{'}
\end{equation} -энергия относительно основного состояния;
\\*
Тогда:
\begin{equation} \label{eq:aperp} % the label is used to reference the equation
\mu_a = Tln(\frac{\lambda{_a}^3 n_a}{z_{in}*exp(- \frac{E_{a}^{0}}{T})})
\end{equation}
\begin{equation} \label{eq:aperp} % the label is used to reference the equation
\mu_{a}=\mu_{i} + \mu_{e} 
\end{equation}
\begin{equation} \label{eq:aperp} % the label is used to reference the equation
\mu_{a}=Tln{\frac{n_a \lambda_{a}^{3}}{e^{\frac{-E_{a}^{0}}{T}}z_{a}^{'}}} 
\end{equation}
\begin{equation} \label{eq:aperp} % the label is used to reference the equation
\mu_{i}=Tln{\frac{n_i \lambda_{i}^{3}}{e^{\frac{-E_{i}^{0}}{T}}z_{i}^{'}}}
\end{equation}
\begin{equation} \label{eq:aperp} % the label is used to reference the equation
\mu_{e}=Tln{\frac{n_e \lambda_{e}^{3}}{e^{\frac{-E_{e}^{0}}{T}}z_{e}^{'}}}
\end{equation}
-подставим в элементарнтарный баланс энергий:
\begin{equation} \label{eq:aperp} % the label is used to reference the equation
T[ln{\frac{n_a \lambda_{a}^{3}}{e^{\frac{-E_{a}^{0}}{T}}z_{a}^{'}}- ln{\frac{n_i \lambda_{i}^{3}}{e^{\frac{-E_{i}^{0}}{T}}z_{i}^{'}}}} - ln{\frac{n_e \lambda_{e}^{3}}{e^{\frac{-E_{e}^{0}}{T}}z_{e}^{'}}} ]=0 
\end{equation}
\begin{equation} \label{eq:aperp} % the label is used to reference the equation
ln[\frac{n_a \lambda_{a}^{3}* {z}_{i}^{'}{z}_{e}^{'}e^{\frac{-E_{i}^{0}-E_{e}^{0}}{T}}}{e^{\frac{-E_{a}^{0}}{T}}z_{a}^{'}* n_i n_e(\lambda_i \lambda_e)^3}]=0 
\end{equation}
\begin{equation} \label{eq:aperp} % the label is used to reference the equation
(\frac{n_a}{n_i n_e}) \frac{z_{i}^{'}z_{e}^{'}}{z_{a}^{'}}e^\frac{E_{a}^{0}-E_{i}^{0}-E_{e}^{0}}{T}(\frac{\lambda_a}{\lambda_i \lambda_e})^3=1 
\end{equation}
Найдем:
\begin{equation} \label{eq:aperp} % the label is used to reference the equation
\frac{n_e n_i}{n_a}=(\frac{\lambda_a}{\lambda_i \lambda_e})^3\frac{z_{i}^{'}z_{e}^{'}}{z_{a}^{'}}e^{-\frac{I}{T}} 
\end{equation}

\begin{equation} \label{eq:aperp} % the label is used to reference the equation
\lambda_b= \frac{2 \pi \hbar}{\sqrt{2 \pi m T}}
\end{equation}
\begin{equation} \label{eq:aperp} % the label is used to reference the equation
\frac{\lambda_a}{\lambda_i \lambda_e}=\frac{1* \sqrt{m_e} \sqrt{2 \not{m_a} \pi T}}{2 \pi \hbar * \sqrt{\not{m_a}}}= \sqrt{\frac{m_eT}{2 \pi \hbar^2}}
\end{equation}
\begin{equation} \label{eq:aperp} % the label is used to reference the equation
\frac{n_e n_i}{n_a} = (\frac{m_eT}{2 \pi \hbar^2})^{3/2}\frac{z_{i}^{'}z_{e}^{'}}{z_{a}^{'}}e^{-\frac{I}{T}}
\end{equation}
- формула Саха для ионизационного равновесия;
\\*
Введем степень ионизации равную отношению количества ионизированных атомов к общему количеству до ионизации :
\begin{equation} \label{eq:aperp} % the label is used to reference the equation
\alpha= \frac{n_e}{n_e+n_a}
\end{equation}

\begin{equation} \label{eq:aperp} % the label is used to reference the equation
n_e = n_i
\end{equation}

\begin{equation} \label{eq:aperp} % the label is used to reference the equation
n_e + n_i + n_a = n_0,
\end{equation}
где
\begin{equation} \label{eq:aperp} % the label is used to reference the equation
n_0=\frac{P}{T(k_b)}
\end{equation}
- общая концентрация газа;
\begin{equation} \label{eq:aperp} % the label is used to reference the equation
\alpha(n_e+n_a)=n_e
\end{equation}
Выразим концентрации электронов и атомов через степень ионизации и общую концентрацию:

\begin{equation} \label{eq:aperp} % the label is used to reference the equation
n_a=n_0-n_e-n_i
\end{equation}

\begin{equation} \label{eq:aperp} % the label is used to reference the equation
n_e =\alpha(\not{n_e} + n_0 - \not{n_e} - n_i)
\end{equation}
\begin{equation} \label{eq:aperp} % the label is used to reference the equation
n_e = \alpha(n_0 - n_e)
\end{equation}

\begin{equation} \label{eq:aperp} % the label is used to reference the equation
n_e = \frac{\alpha n_0}{1+ \alpha} = \frac{\alpha}{1+\alpha}n_0=n_i
\end{equation}

\begin{equation} \label{eq:aperp} % the label is used to reference the equation
n_a = n_0 - 2n_e = n_0 - \frac{2 \alpha}{1+ \alpha}=n_0(\frac{1 - \alpha}{1 + \alpha })
\end{equation}
В уравнение (29) подставим :
\begin{equation} \label{eq:aperp} % the label is used to reference the equation
\frac{(\frac{\alpha}{1+\alpha})^2n_0^2}{(1-\alpha)(a+\alpha)n_0}=\frac{\alpha^2}{(1-\alpha)(1+\alpha)}n_0=\frac{\alpha^2}{1-\alpha^2}n_0
\end{equation}

\begin{equation} \label{eq:aperp} % the label is used to reference the equation
\frac{\alpha^2}{1-\alpha^2}=(\frac{m_e T}{2 \pi \hbar^2})^{3/2}\frac{2 z_i}{z_a}\frac{T}{P}*e^{-\frac{I}{T}}
\end{equation}
- формула для расчета ионизационного равновесия газа
\section{Покажем, что газ можно считать идеальным}
Радиус Дебая:
\begin{equation}
r_D=\sqrt{\frac{Tk_bE_0}{4\pi n_e e^2}}
\end{equation}
Концентрация электронов:
\begin{equation}
n_e=\frac{P}{kT} \frac{\alpha}{\alpha + 1}
\end{equation}
Условие идеальности:\\*
Г - Параметр неидеальности
\begin{equation}
\Gamma = \frac{e^2}{k_b r_D T}< < 1
\end{equation}

\section{Расчетные данные для Ne}
Потенциал ионизации атома: I = 21.5645 эВ
\\* Электронная конфигурация атома в основном состоянии: 
\\* Статвес основного состояния атома: 1
\\* Электронная конфигурация иона 
\\* Статвес основного состояния иона: 1
\\* Диапазон температур: 3000 - 30 000 K
\\* Диапазон давлений: 0,1 - 1 атм
\\* Константа Больцмана в СИ: $1.38*10^{-23}$ Дж/К
\\* Константа Больцмана в СГСЭ: $1.4*10^{-16}$ эрг/К
\\* Масса электрона: $9.10938356^10^{-31}$ кг
\\* Постоянная планка: $6.582*10^{-16}$ эВ/с

\section{Расчетные данные для He}
Потенциал ионизации атома: I = 24.5874 
\\* Электронная конфигурация атома в основном состоянии: 
\\* Статвес основного состояния атома: 1
\\* Электронная конфигурация иона 
\\* Статвес основного состояния иона: 1
\\* Диапазон температур: 3000 - 30 000 K
\\* Диапазон давлений: 0,1 - 1 атм
\\* Константа Больцмана в СИ: $1.38*10^{-23}$ Дж/К
\\* Константа Больцмана в СГСЭ: $1.4*10^{-16}$ эрг/К
\\* Масса электрона: $9.10938356^10^{-31}$ кг
\\* Постоянная планка: $6.582*10^{-16}$ эВ/с

\section{Вывод}
При численной оценке получаенно, что параметр неидеальности много меньше 1. Поэтому в расчетах применялись формулы без учета параметра неидеальности. Также численный расчет подтвердил, что в среднем степень ионизации равна 0,5 при тепературе равной одной десятой потенциала ионизации.

%++++++++++++++++++++++++++++++++++++++++
% References section will be created automatically 
% with inclusion of "thebibliography" environment
% as it shown below. See text starting with line
% \begin{thebibliography}{99}
% Note: with this approach it is YOUR responsibility to put them in order
% of appearance.

\begin{thebibliography}{99}

\bibitem{melissinos}
A.~C. Melissinos and J. Napolitano, \textit{Experiments in Modern Physics},
(Academic Press, New York, 2003).

\bibitem{Cyr}
N.\ Cyr, M.\ T$\hat{e}$tu, and M.\ Breton,
% "All-optical microwave frequency standard: a proposal,"
IEEE Trans.\ Instrum.\ Meas.\ \textbf{42}, 640 (1993).

\bibitem{Wiki} \emph{Expected value},  available at
\texttt{http://en.wikipedia.org/wiki/Expected\_value}.

\end{thebibliography}


\end{document}